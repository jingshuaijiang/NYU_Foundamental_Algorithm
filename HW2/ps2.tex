% me=0 student solutions (ps file), me=1 - my solutions (sol file),
% me=2 - assignment (hw file)
\def\me{0} \def\num{2} %homework number

\def\due{5 pm on Thursday, September 19} %due date

\def\course{CSCI-GA.1170-001/002 Fundamental Algorithms} 
%course name, changed only once

% **** INSERT YOUR NAME HERE ****
\def\name{Jingshuai Jiang}

% **** INSERT YOUR NETID HERE ****
\def\netid{jj2903}

% **** INSERT NETIDs OF YOUR COLLABORATORS HERE ****
\def\collabs{NetID1, NetID2}


\iffalse

INSTRUCTIONS: replace # by the homework number.  (if this is not
ps#.tex, use the right file name)

Clip out the ********* INSERT HERE ********* bits below and insert
appropriate LaTeX code.  There is a section below for student macros.
It is not recommended to change any other parts of the code.


\fi
%

\documentclass[11pt]{article}


% ==== Packages ====
\usepackage{amsfonts,amsmath}
\usepackage{latexsym}
\usepackage{fullpage}
\usepackage{graphicx}
\usepackage{listings}
\usepackage{graphicx}
\usepackage{float}
\usepackage[bottom]{footmisc}


% \setlength{\oddsidemargin}{.0in} \setlength{\evensidemargin}{.0in}
% \setlength{\textwidth}{6.5in} \setlength{\topmargin}{-0.4in}
\setlength{\footskip}{1in} \setlength{\textheight}{8.5in}

\newcommand{\handout}[5]{
\renewcommand{\thepage}{#1, Page \arabic{page}}
  \noindent
  \begin{center}
    \framebox{ \vbox{ \hbox to 5.78in { {\bf \course} \hfill #2 }
        \vspace{4mm} \hbox to 5.78in { {\Large \hfill #5 \hfill} }
        \vspace{2mm} \hbox to 5.78in { {\it #3 \hfill #4} }
        \ifnum\me=0
        \vspace{2mm} \hbox to 5.78in { {\it Collaborators: \collabs
            \hfill} }
        \fi
      } }
  \end{center}
  \vspace*{4mm}
}

\newcounter{pppp}
\newcommand{\prob}{\arabic{pppp}} %problem number
\newcommand{\increase}{\addtocounter{pppp}{1}} %problem number

% Arguments: Title, Number of Points
\newcommand{\newproblem}[2]{
  \ifnum\me=0
    \ifnum\prob>0 \newpage \fi
    \increase
    \setcounter{page}{1}
    \handout{\name{} (\netid), Homework \num, Problem \arabic{pppp}}
    {\today}{Name: \name{} (\netid)}{Due: \due}
    {Solutions to Problem \prob\ of Homework \num\ (#2)}
  \else
    \increase
    \section*{Problem \num-\prob~(#1) \hfill {#2}}
  \fi
}

% \newcommand{\newproblem}[2]{\increase
% \section*{Problem \num-\prob~(#1) \hfill {#2}}
% }

\def\squarebox#1{\hbox to #1{\hfill\vbox to #1{\vfill}}}
\def\qed{\hspace*{\fill}
  \vbox{\hrule\hbox{\vrule\squarebox{.667em}\vrule}\hrule}}
\newenvironment{solution}{\begin{trivlist}\item[]{\bf Solution:}}
  {\qed \end{trivlist}}
\newenvironment{solsketch}{\begin{trivlist}\item[]{\bf Solution
      Sketch:}} {\qed \end{trivlist}}
\newenvironment{code}{\begin{tabbing}
    12345\=12345\=12345\=12345\=12345\=12345\=12345\=12345\= \kill }
  {\end{tabbing}}

%\newcommand{\eqref}[1]{Equation~(\ref{eq:#1})}

\newcommand{\hint}[1]{({\bf Hint}: {#1})}
% Put more macros here, as needed.
\newcommand{\room}{\medskip\ni}
\newcommand{\brak}[1]{\langle #1 \rangle}
\newcommand{\bit}[1]{\{0,1\}^{#1}}
\newcommand{\zo}{\{0,1\}}
\newcommand{\C}{{\cal C}}

\newcommand{\nin}{\not\in}
\newcommand{\set}[1]{\{#1\}}
\renewcommand{\ni}{\noindent}
\renewcommand{\gets}{\leftarrow}
\renewcommand{\to}{\rightarrow}
\newcommand{\assign}{:=}

\newcommand{\AND}{\wedge}
\newcommand{\OR}{\vee}

\newcommand{\For}{\mbox{\bf for }}
\newcommand{\To}{\mbox{\bf to }}
\newcommand{\Do}{\mbox{\bf do }}
\newcommand{\If}{\mbox{\bf if }}
\newcommand{\Then}{\mbox{\bf then }}
\newcommand{\Else}{\mbox{\bf else }}
\newcommand{\While}{\mbox{\bf while }}
\newcommand{\Repeat}{\mbox{\bf repeat }}
\newcommand{\Until}{\mbox{\bf until }}
\newcommand{\Return}{\mbox{\bf return }}
\newcommand{\Halt}{\mbox{\bf halt }}
\newcommand{\Swap}{\mbox{\bf swap }}
\newcommand{\Ex}[2]{\textrm{exchange } #1 \textrm{ with } #2}



\begin{document}

\ifnum\me=0

% Collaborators (on a per task basis):
%
% Task 1: *********** INSERT COLLABORATORS HERE *********** 
% Task 2: *********** INSERT COLLABORATORS HERE *********** 
% etc.
%

\fi

\ifnum\me=1

\handout{PS \num}{\today}{Lecturer: Yevgeniy Dodis}{Due: \due}
{Solution {\em Sketches} to Problem Set \num}

\fi

\ifnum\me=2

\handout{PS \num}{\today}{Lecturer: Yevgeniy Dodis}{Due: \due}{Problem
  Set \num}

\fi
\newproblem{Different Methods for Recurrences}{14 points}

Consider the recurrence $T(n) = 8T(n/4) + n$ with initial condition
$T(1)=1$.

\begin{itemize}

\item[(a)] (2 points) Solve it asymptotically using the ``master theorem''.

\ifnum\me<2
\begin{solution}    according to the master theorem, Because the $a=8$,$b=4$,so $$f(n)=n<n^{\log_4 8}$$ so$$T(n)=\Theta(n^{\log_4 8})=\Theta(n^\frac{3}{2})$$
   
\end{solution}
\fi

\item[(b)] (4 points) Solve it by the ``guess-then-verify method''. Namely, guess a
function $g(n)$ --- presumably solving part (a) will give you a good
guess --- and argue by induction that for all values of $n$ we have
$T(n) \le g(n)$. What is the ``smallest'' $g(n)$ for which your
inductive proof works?

\ifnum\me<2
\begin{solution}   guess that $T(n) \le c\cdot n^2$ then $T(1)=1 \le c$,
  \\[10pt]then for $n \ge 2$ $$T(n)=8T(\frac{n}{4})+n \le 8\cdot c\cdot (\frac{n}{4})^2+n=c\cdot \frac{n^2}{2}+n \le c\cdot n^2$$ 
  then we get $$\frac{n^2}{2}\cdot c-n \ge 0$$ 
  \\[10pt] when $c = 1 $ the equation is true for both T(1) and T(n) $n \ge 2$
  \\[10pt]then the ``smallest'' $g(n)$ should be $n^2$
   \end{solution}
\fi

\item[(c)] (4 points) Solve it by the ``recursive tree method''. Namely, draw the
full recursive tree for this recurrence, and sum up all the value to
get the final time estimate. Again, try to be as precise as you can
(i.e., asymptotic answer is OK, but would be nice if you preserve a
``leading constant'' as well).

\ifnum\me<2
\begin{solution}   
  \begin{figure}[H]

    \centering
    \includegraphics[scale=0.2]{tree.png}
    \caption{this is a recurrsive tree}
    \label{fig:label}
    \end{figure}
    $$(2^0+2^1+2^2+........+2^{\log_4 n})\cdot n=(2\cdot\sqrt{n}-1)\cdot n = 2 \cdot n^\frac{3}{2}-n=\Theta(n^\frac{3}{2})$$
  \end{solution}
\fi

\item[(d)] (4 points) Solve it {\em precisely} using the ``domain-range
substitution'' technique. Namely, make several changes of variables
until you get a basic recurrence of the form $S(k) = S(k-1) + f(k)$
for some $f$, and then compute the answer from there. Make sure you
carefully maintain the correct initial condition.

\ifnum\me<2
\begin{solution}   
  \\[10pt] let $n=4^k$ then $$T(n)=T(4^k)=8\cdot T(4^{k-1})+4^k$$ let $$T(4^k)=S(k)$$ then $$\frac{S(k)}{8^k}=\frac{S(k-1)}{8^{k-1}}+\frac{4^k}{8^k}$$ then let$$ \frac{S(k)}{8^k}=g(k)$$ $$g(k)=g(k-1)+{\frac{1}{2}}^k$$
  then we get $$g(k)={\frac{1}{2}}^0+.....+{\frac{1}{2}}^k=2-{\frac{1}{2}}^k$$ and we get$$S(k)=2\cdot (4\cdot 2)^k-4^k$$ and we get $$T(n)=2\cdot n\sqrt{n}-n=\Theta(n^\frac{3}{2})$$
   \end{solution}
\fi
\end{itemize}



\newproblem{Functionality vs. Running Time}{10 points}


\noindent
Consider the following recursive procedure.

\medskip

\begin{code}
\>{\sc Bla}$(n)$:\\
\> \> \If $n=1$ \Then \Return $1$\\
\> \> \Else \Return $\mbox{\sc Bla}(n-1) + \mbox{\sc Bla}(n-1) +
\mbox{\sc Bla}(n-1)$
\end{code}

\begin{itemize}

\item[(a)] (3 points) What function of $n$ does {\sc Bla} compute?

\ifnum\me<2
\begin{solution}  
  \begin{equation}
      BLA(n)=
     \begin{cases}
     1&\mbox{if $n$ = 1}\\
     BLA(n-1)+BLA(n-1)+BLA(n-1)&\mbox{if $n>=1$}
     \end{cases}
    \end{equation}
according to the function of BLA(n) it equals to $3\cdot BLA(n-1)$,and $3^2\cdot BLA(n-2)$ by doing this recursively it equals to $$3^{n-1}$$   \end{solution}
\fi

\item[(b)] (4 points) What is the running time $T(n)$ of {\sc Bla}?

\ifnum\me<2
\begin{solution}   
  \\[10pt]according to the expression of BLA, we get $$T(n) = 3T(n-1)$$
  by using the range-domain-substitution we get $$g(n) = \frac{T(n)}{3^n} =\frac{T(n-1)}{3^{n-1}}$$
  and we get$$T(n) = 3^{n-1}=\Theta(3^n)$$
 \end{solution}
\fi

\item[(c)] (3 points) How do the answers to (a) and (b) change if we replace the
last line by\\ ``\Else \Return $3\cdot \mbox{\sc Bla}(n-1)$''?

\ifnum\me<2
\begin{solution}   The answer to (a) will remain the same because $BLA(n)= 3^n$ 
\\[10pt]  but the answer to (b) will change to $$T(n)=\Theta(n)$$ because the algorithm will only have to caculate BLA(1) to BLA(n) for only one time of each, and the whole process will have n units to caculate.That makes the final time to $\Theta(n)$
 \end{solution}
\fi

\end{itemize}
\newproblem{Fun with Recurrences}{17 Points}

\begin{itemize}
\item[(a)](4 pts) Consider the following recurrence 
$$T(n) = \sqrt{n} \, T(\sqrt{n}) + n .$$ Solve for $T(n)$ by domain-range substitution. (\textbf{Hint}: divide both parts by "something").

\ifnum\me<2
\begin{solution}  
  \\[10pt]  $$\frac{T(n)}{n} = \frac{T(\sqrt{n})}{\sqrt{n}}+1$$ then $$S(n)=S(\sqrt{n})+1=S(2)+1+.....+1=S(2)+loglogn=1+loglogn$$ 
  then $$T(n)=n+nloglogn=\Theta(nloglogn)$$
   \end{solution}
\fi


\item[(b)](4 pts) Consider the following recurrence 
$$T(n) = 2 T(n/2) + n/\log n .$$ Solve for $T(n)$ by domain-range substitution.
\ifnum\me<2
\begin{solution}
\\[10pt]    let$$n=2^k$$   then $$T(2^k)=2T(2^{k-1})+\frac{2^k}{k}$$ then $$\frac{T(2^k)}{2^k}=\frac{T(2^{k-1})}{2^{k-1}}+\frac{1}{k}$$
then we get(when k approaches infinite) $$S(k)=S(k-1)+\frac{1}{k}=1+.....+\frac{1}{k}=lnk+c$$ then $$T(2^k)=2^klnk+c\cdot 2^k$$ and $$T(n)=n\cdot lnlogn+c\cdot n=\Theta(nlnlogn)$$
\end{solution}
\fi

\item[(c)] (4 points) $T(n)  = T(n/2) + \log n$.

\ifnum\me<2
\begin{solution}   
\\[10pt]let$$n=2^k$$ then $$T(2^k)=T(2^{k-1})+k$$ let $T(2^k)=S(k)$ then $$S(k)=S(k-1)+k $$ and we get $$S(k)=1+.......+k=\frac{(1+k)\cdot k}{2}$$ 
then $$T(n)=\frac{logn+log^2n}{2}=\Theta(log^2n)=\Theta((log n)^2)$$  Ps.the finally two are the same expression.And the first is widely used in China \end{solution}
\fi
\item[(d)] (5 Points) $T(n)=3T(n-1)-2T(n-2),n>2, T(2)=1, T(1)=0$.
\ifnum\me<2
\begin{solution}   
\\[10pt] by moving one $T(n-1)$ to the left of the equation we get $$T(n)-T(n-1) = 2(T(n-1)-T(n-2))$$ let $$T(n)-T(n-1)=S(n)$$ then we get $$S(n)=2\cdot S(n-1)$$ because $T(2) =1, T(1)=0$ we can get that $S(2)=1$ then $$S(n)=2^{n-2}\cdot S(2)=2^{n-2}$$ so $$T(n)-T(n-1)=2^{n-2}$$ 
then $$T(n)=1+....+2^{n-2} = 2^{n-1}-1=\Theta(2^n)$$
\end{solution}
\fi

\end{itemize}

\end{document}


