% me=0 student solutions (ps file), me=1 - my solutions (sol file),
% me=2 - assignment (hw file)
\def\me{0} \def\num{10} %homework number

\def\due{5 pm on Wednesday, November 27th} %due date

\def\course{CSCI-GA.1170-001/002 Fundamental Algorithms} 
%course name, changed only once

% **** INSERT YOUR NAME HERE ****
\def\name{jingshuai jiang}

% **** INSERT YOUR NETID HERE ****
\def\netid{jj2903}

% **** INSERT NETIDs OF YOUR COLLABORATORS HERE ****
\def\collabs{NetID1, NetID2}


\iffalse

INSTRUCTIONS: replace # by the homework number.  (if this is not
ps#.tex, use the right file name)

Clip out the ********* INSERT HERE ********* bits below and insert
appropriate LaTeX code.  There is a section below for student macros.
It is not recommended to change any other parts of the code.


\fi
%

\documentclass[11pt]{article}


% ==== Packages ====
\usepackage{amsfonts,amsmath}
\usepackage{latexsym}
\usepackage{fullpage}
\usepackage{subcaption}
\usepackage{graphicx}
\usepackage{tikz}
\usepackage[bottom]{footmisc}


% \setlength{\oddsidemargin}{.0in} \setlength{\evensidemargin}{.0in}
% \setlength{\textwidth}{6.5in} \setlength{\topmargin}{-0.4in}
\setlength{\footskip}{1in} \setlength{\textheight}{8.5in}

\newcommand{\handout}[5]{
\renewcommand{\thepage}{#1, Page \arabic{page}}
  \noindent
  \begin{center}
    \framebox{ \vbox{ \hbox to 5.78in { {\bf \course} \hfill #2 }
        \vspace{4mm} \hbox to 5.78in { {\Large \hfill #5 \hfill} }
        \vspace{2mm} \hbox to 5.78in { {\it #3 \hfill #4} }
        \ifnum\me=0
        \vspace{2mm} \hbox to 5.78in { {\it Collaborators: \collabs
            \hfill} }
        \fi
      } }
  \end{center}
  \vspace*{4mm}
}

\newcounter{pppp}
\newcommand{\prob}{\arabic{pppp}} %problem number
\newcommand{\increase}{\addtocounter{pppp}{1}} %problem number

% Arguments: Title, Number of Points
\newcommand{\newproblem}[2]{
  \ifnum\me=0
    \ifnum\prob>0 \newpage \fi
    \increase
    \setcounter{page}{1}
    \handout{\name{} (\netid), Homework \num, Problem \arabic{pppp}}
    {\today}{Name: \name{} (\netid)}{Due: \due}
    {Solutions to Problem \prob\ of Homework \num\ (#2)}
  \else
    \increase
    \section*{Problem \num-\prob~(#1) \hfill {#2}}
  \fi
}

% \newcommand{\newproblem}[2]{\increase
% \section*{Problem \num-\prob~(#1) \hfill {#2}}
% }

\def\squarebox#1{\hbox to #1{\hfill\vbox to #1{\vfill}}}
\def\qed{\hspace*{\fill}
  \vbox{\hrule\hbox{\vrule\squarebox{.667em}\vrule}\hrule}}
\newenvironment{solution}{\begin{trivlist}\item[]{\bf Solution:}}
  {\qed \end{trivlist}}
\newenvironment{solsketch}{\begin{trivlist}\item[]{\bf Solution
      Sketch:}} {\qed \end{trivlist}}
\newenvironment{code}{\begin{tabbing}
    12345\=12345\=12345\=12345\=12345\=12345\=12345\=12345\= \kill }
  {\end{tabbing}}

%\newcommand{\eqref}[1]{Equation~(\ref{eq:#1})}

\newcommand{\hint}[1]{({\bf Hint}: {#1})}
% Put more macros here, as needed.
\newcommand{\room}{\medskip\ni}
\newcommand{\brak}[1]{\langle #1 \rangle}
\newcommand{\bit}[1]{\{0,1\}^{#1}}
\newcommand{\zo}{\{0,1\}}
\newcommand{\C}{{\cal C}}

\newcommand{\nin}{\not\in}
\newcommand{\set}[1]{\{#1\}}
\renewcommand{\ni}{\noindent}
\renewcommand{\gets}{\leftarrow}
\renewcommand{\to}{\rightarrow}
\newcommand{\assign}{:=}

\newcommand{\AND}{\wedge}
\newcommand{\OR}{\vee}

\newcommand{\For}{\mbox{\bf for }}
\newcommand{\To}{\mbox{\bf to }}
\newcommand{\DownTo}{\mbox{\bf downto }}
\newcommand{\Do}{\mbox{\bf do }}
\newcommand{\If}{\mbox{\bf if }}
\newcommand{\Then}{\mbox{\bf then }}
\newcommand{\Else}{\mbox{\bf else }}
\newcommand{\While}{\mbox{\bf while }}
\newcommand{\Repeat}{\mbox{\bf repeat }}
\newcommand{\Until}{\mbox{\bf until }}
\newcommand{\Return}{\mbox{\bf return }}
\newcommand{\Halt}{\mbox{\bf halt }}
\newcommand{\Swap}{\mbox{\bf swap }}
\newcommand{\Ex}[2]{\textrm{exchange } #1 \textrm{ with } #2}
\newcommand{\Nil}{\mbox{\bf nil }}
\newcommand{\False}{\textsc{False}}
\newcommand{\True}{\textsc{True}}
\begin{document}

\ifnum\me=0

% Collaborators (on a per task basis):
%
% Task 1: *********** INSERT COLLABORATORS HERE *********** 
% Task 2: *********** INSERT COLLABORATORS HERE *********** 
% etc.
%

\fi

\ifnum\me=1

\handout{PS \num}{\today}{Lecturer: Yevgeniy Dodis}{Due: \due}
{Solution {\em Sketches} to Problem Set \num}

\fi

\ifnum\me=2

\handout{PS \num}{\today}{Lecturer: Yevgeniy Dodis}{Due: \due}{Problem
  Set \num}

\fi
\newproblem{Sokoban}{11 Points}

\newcommand{\init}{I}
\newcommand{\target}{T}
\newcommand{\wa}{w}

\noindent
An $n \times n$-grid is a graph $G$ with $n^2$ vertices
$\{(i,j) \mid 1 \leq i,j \leq n\}$ and edges such that two vertices
$u$ and $v$ are connected if and only if they are at distance $1$.

The game of Sokoban is formalized as follows:%
\footnote{Feel free to watch the animation on Wikipedia.} %
Let $S = (V,E)$ be a subgraph of the $n \times n$-grid, i.e., $S$ can
be obtained by deleting all vertices of the grid not contained in $V$
and the edges adjacent to them.  Imagine that at the onset of the
game, some vertices $\init \subseteq V$ on subgrid $S$ have a crate on
them and that there is one special vertex $\wa \in V \setminus \init$
where Waldo stands.  Moreover, there is a set of target positions
$\target \subseteq V$ with $k := |\target| = |\init|$ for the crates.
Waldo's goal is to move the crates in such a way that in the end there
is a crate on each vertex in $\target$.

At any point, Waldo can move freely about the subgrid vertices not
covered by crates.  He can move an adjacent crate by standing behind
it and pushing by one vertex, provided there is no other crate in the
way.  For example, if Waldo is at $(i,j) \in V$, there is a crate at
$(i+1,j) \in V$ and no crate at $(i+2,j)$, he can push the crate to
$(i+2,j) \in V$; Waldo ends up at $(i+1,j)$ in such a case.

\begin{itemize}
  \item [(a)] (3 points) Show that the number of different states of
  this game is at most $\left(n^2\right)^{k+1}$.

  \ifnum\me<2
\begin{solution}   
There are k caters and one people. Then the man can have $n^2$ positions. The k caters should have $(n^2-1)\cdot(n^2-2)...\cdot (n^2-k)$ positions. Then the total state should be $n^2\cdot (n^2-1)\cdot(n^2-2)...\cdot (n^2-k) \le \left(n^2\right)^{k+1}$
This equation will be achieved when $k=0$.\\[10pt]
So the number of different states of this game is at most $\left(n^2\right)^{k+1}$.
\end{solution}
  \fi

  \item [(b)] (3 points) Describe a winning state of this game and
  argue that there are at most $n^2 k!$ of them.

  \ifnum\me<2
\begin{solution}   
In this game, the wining state of this game will be : There is a crate on each vertex in T,this permutations will have $k!$. 
And the man can be in any other vertex, which is $n^2-k$.So the toal number of winning state should be $(n^2-k)k! \le n^2k!$.
\end{solution}
  \fi

  \item [(c)] (5 points) Devise an
  $O\left(\left(n^2\right)^{k+1}\right)$-algorithm that on input
  $S = (V,E)$ figures out the fastest way for Waldo to solve this
  problem (i.e., to move the crates from their initial positions $I$
  to their target positions $T$) or reports that there is no solution.

  Argue why your algorithm works and achieves the desired running
  time.

  \hint{Using one of the algorithms you have seen in class, explore a
    state graph corresponding to Sokoban, in which the vertices are
    the various states and two states are connected if and only if in
    one move of the game (either Waldo only or Waldo and a crate
    change positions) one can get from one to the other.}

  \ifnum\me<2
\begin{solution}   
I will use the BFS algorithm.  BFS will find the minimal distance from the start point to any other vertices. And we can use this to achive this by turning this whole process into a BFS search. This whole process will being like this: 1. Think of this problem as finding the minimum distance from the initialized state to the winning state. 2. Get the initialize state of both
waldo and crates and mark this state as the starting point. 3. all states $u.d = \infty$ and $u.\pi = nill$. 4. and all of the state(vertices) will have whilte,gray,black states as shown in our books.
5. use the BFS algorithm start from the starting point and keep looking for the winning state. IF we get the winning state, the whole process will be finished.\\[10pt]
Here is the sudocode.

\begin{code}

    1 \> {\sc Sokoban}$(G,s)$\\
    2 \> \> s.color = gray,s.d = 0,$s.\pi = nil$, Q is empty\\
    3 \> \> Enqueue(Q,s)\\
    4 \> \> while Q is not empty\\
    5 \> \> \>u $\leftarrow$ Dequeue(Q) \\
    6 \> \> \>\For all of v adj to U\\
    7 \> \> \>\> \If v.color = white\\
    8 \>\>\>\>\> v.color = gray\\
    9 \>\>\>\>\> \If v is winningstate \Then \Return v\\
    10 \>\>\>\>\> v.d = u.d+1\\
    11 \>\>\>\>\> $v.\pi = u$\\
    12 \>\>\>\>\> Enqueue(Q,v)\\
    13 \>\>\>\> u.color = black\\
    14\>\> return nil\\
    
\end{code}

Run time: Using BFS to traversing the whole graph will take O(m+n), in this algorithm we have $O((n^2)^{k+1})$ vertices, and each vertices will only have constant edges with the surrounding vertices. This means we have  $O(c(n^2)^{k+1})$ edges. Then in this way the
whole time will be $O((n^2)^{k+1})$ .
\end{solution}
  \fi
\end{itemize}

\newproblem{Avoiding Double Infections}{11 points}

\noindent
You are given a directed graph $G=(V,E)$ on $n$ nodes and $m$ edges,
where the node set $V = A\cup B$ consists of two disjoint subsets $A$
and $B$ of sizes $n_1$ and $n_2$ (so $n=n_1+n_2$). Nodes in $A$ are
``healthy'', while nodes in $B$ are ``infected''. Given a source $s\in
A$, your goal is to compute the shortest distance from $s$ to every
other healthy node which can pass through {\em at most one} infected
node (i.e., if the path from $s$ to $v$ contains at most one infected
$u$, this is OK, but if it contains two or more, this path is not
allowed when computing the shortest distance).

The problem is a bit tricky, so I will help you. Define the following
{\em directed} graph $G' = (V',E')$ on $2n_1+n_2$ nodes. The vertex
set of $V'$ of $G'$ is $V'= A_1\cup B \cup A_2$, where $A_1$ and $A_2$
are two copies of healthy nodes $A$. Two nodes in $A_1$ are connected
in $G'$ if and only if they are connected in $G$, and the same between
two nodes in $A_2$. The nodes in $B$ are more interesting. For every
original incoming edge $(a,b)\in E$, where $a\in A$ and $b\in B$, we
will put an edge $(a_1,b)$ in $E'$, where $a_1$ is the copy of $a$ in
$A_1$. Similarly, for every original outgoing edge $(b,a)\in E$, where
$a\in A$ and $b\in B$, we will put an edge $(b,a_2)$ in $E'$, where
$a_2$ is the copy of $a$ in $A_2$.

\begin{itemize}

\item[(a)] (4 points) Let $n',m'$ be the number of vertices and edges in
$G'$. Show that $n'\le 2n$ and $m'\le 2m$.

\ifnum\me<2
\begin{solution}   
  Since the vertex set of $V'$ of $G'$ is $V'= A_1\cup B \cup A_2$. And $A_2$ is a copy of A. Then the number of vertices in $G'$ will be $n_1+n_2+n_1 = n+n_1$. And apparently the number of 
  vertices in A $n_1$ is smalller than or equal to $n$. Then $n+n_1 \le n+n = 2n$\\[10pt]
  It is the same with the edges. The edges og $G$ can be split into four parts$E_{aa},E_{bb},E_{ab},E_{ba}$ and $E_{aa}+E_{bb}+E_{ab}+E_{ba} = m$ and $m'=E_{G'} = E_{aa}+E_{ab}+E_{ba'}+E_{a'a'} = m-E_{bb}+E_{aa} \le 2m$
\end{solution}
\fi

\item[(b)] (4 points) Recall our original problem of computing the required
shortest distance in $G$ from $s$ to every other healthy node $a\in A$
which can pass through {\em at most one} infected node $b\in B$. Call
this distance $a[dis]$. Let $s_1$ and $s_2$ be the two copies of $s$
in $G'$. Using one ``appropriate'' BFS call on $G'$, show how to
compute the values $a[dis]$. Specifically, say what is the starting
node (call it $s'$) of your BFS call in $G'$. Also, after your BFS
call computed shortest distances $v'.d$ from $s'$ to $v'$, for every
$v'\in V'$, show how to compute the desired values $a[dis]$ for the
problem at hand (i.e., write an explicit formula for $a[dis]$ using
appropriate $v'.d$ values). Justify your algorithm.

\ifnum\me<2
\begin{solution}
The starting node $s'$ should be $s_1$ \\[10pt]
Since there are no links in the new graph between nodes B. Then all the nodes of a' linked to this $s'$ will all be satistied points. 
Then in this case we just need a BFS of all the node from the starting point. And we will compare the distance of just using a to a and the distance of a to b to a'
And this for every $v$ as healthy nodes the $a[dis] = min(v.d,v'.d)$
\end{solution}
\fi

\item[(c)](3 points) Show that the running time of your procedure is $O(m+n)$.

\ifnum\me<2
\begin{solution}   
By using BFS of a tree, which has at most 2n nodes and 2m edges, all the nodes will be added to the queue for only one time and all the edges will be visited for one time.The total running time will be $O(2m+2n)$ which is $O(m+n)$
\end{solution}
\fi


\end{itemize}
\newproblem{DFS in place of BFS}{12 points}

\noindent
Assume you want to compute shortest distances $\delta(v)$ from a source $s$ to all nodes $v$ of a graph $G$. Normally, you would run {\sc{BFS}}$(s)$ and have $d(s)=\delta(s)$ at the end. Imagine that instead you run the following modification of {\sc{DFS-Visit}}$(v)$:
\begin{itemize}
\item You initialize $d(s)=0$ and $d(v)=\infty$ for all $v\neq s$.
\item For every {\em tree} edge $(u,v)$ encountered by {\sc{DFS-Visit}}, you set $d(v)=d(u)+1$ (just like {\sc{BFS}}).
\end{itemize}

\begin{itemize}

\item[(a)] (1 point) What is the answer produced by this algorithm on a complete graph $K_n$, where $(u,v)\in E$ for all $(u,v)$?

\ifnum\me<2
\begin{solution}   
The distance will be 0,1,2,...,n-1
\end{solution}
\fi

\item[(b)] (2 points) Give an example of a graph $G$ with at most $2n$ edges, and an ordering of its edges in the adjacency list, where the algorithm above would still produce the same (bogus) answer as in part (a)? 

\ifnum\me<2
\begin{solution}   

\end{solution}
\fi

\item[(c)] (2 points) Give an example of a graph $G$ with $\Omega(n^2)$ edges and $\delta(v)< \infty$ for all $v$, and an ordering of its edges in the adjacency list, where the algorithm above would not only produce a correct answer for all $n$ vertices $v$, but even produce exactly the same BFS tree as the {\sc{BFS}} itself.

\ifnum\me<2
\begin{solution}   INSERT YOUR SOLUTION HERE   \end{solution}
\fi

\item[(d)] (4 points) Recall, a forward edge $(u,v)$ connects $u$ to some (not immediate) descendant $v$ of $u$ in the DFS forest. For each of the following assertions, either prove it is correct, or give a counter-example. 
\begin{itemize}
\item If the algorithm above is correct on all nodes $v$, then {\sc{DFS-Visit}}$(s)$ did not encounter any forward edges.
\item If the algorithm above is incorrect on at least one node $v$, then {\sc{DFS-Visit}}$(s)$ encountered at least one forward edge.
\end{itemize}

\ifnum\me<2
\begin{solution}   
The first is True. Assume we have a forward edge and the algorithm is correct in all the node. Assume the distance from s to $v_1$ is $d_1$, the distance from s to $v_2$ is $d_2$. Since $v_2$ is a decentant of $v_1$ (not immediate) then we have 
$d_2 > d_1+1$. However we have this forward edge, which means we can write $d_2 = d_1+1$. Then this $d_2$ is wrong, then we have a contradict here. This means we do not encounter any forward edges.\\[10pt]

\end{solution}
\fi

\item[(e)] (3 points) Assume now we modify our procedure as follows: instead of setting
$d(v) = d(u)+1$ only for tree edges $(u,v)$ (color WHITE), we right away set 
$d(v) = \min(d(v),d(u)+1)$ for every discovered edge $(u,v)$ irrespective of the color of $v$. 
Give an example of a graph $G$, and an ordering of its edges in the adjacency list, where this algorithm is still wrong.

\ifnum\me<2
\begin{solution}   

\end{solution}
\fi

\end{itemize}


\end{document}


