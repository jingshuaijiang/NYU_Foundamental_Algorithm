% me=0 student solutions (ps file), me=1 - my solutions (sol file),
% me=2 - assignment (hw file)
\def\me{0} \def\num{1} %homework number

\def\due{5 pm on Thursday, September 12} %due date

\def\course{CSCI-GA.1170-001/002 Fundamental Algorithms} 
%course name, changed only once

% **** INSERT YOUR NAME HERE ****
\def\name{Jingshuai Jiang}

% **** INSERT YOUR NETID HERE ****
\def\netid{jj2903}

% **** INSERT NETIDs OF YOUR COLLABORATORS HERE ****
\def\collabs{NetID1, NetID2}


\iffalse

INSTRUCTIONS: replace # by the homework number.  (if this is not
ps#.tex, use the right file name)

Clip out the ********* INSERT HERE ********* bits below and insert
appropriate LaTeX code.  There is a section below for student macros.
It is not recommended to change any other parts of the code.


\fi
%

\documentclass[11pt]{article}


% ==== Packages ====
\usepackage{amsfonts}
\usepackage{latexsym}
\usepackage{fullpage}

% \setlength{\oddsidemargin}{.0in} \setlength{\evensidemargin}{.0in}
% \setlength{\textwidth}{6.5in} \setlength{\topmargin}{-0.4in}
\setlength{\footskip}{1in} \setlength{\textheight}{8.5in}

\newcommand{\handout}[5]{
\renewcommand{\thepage}{#1, Page \arabic{page}}
  \noindent
  \begin{center}
    \framebox{ \vbox{ \hbox to 5.78in { {\bf \course} \hfill #2 }
        \vspace{4mm} \hbox to 5.78in { {\Large \hfill #5 \hfill} }
        \vspace{2mm} \hbox to 5.78in { {\it #3 \hfill #4} }
        \ifnum\me=0
        \vspace{2mm} \hbox to 5.78in { {\it Collaborators: \collabs
            \hfill} }
        \fi
      } }
  \end{center}
  \vspace*{4mm}
}

\newcounter{pppp}
\newcommand{\prob}{\arabic{pppp}} %problem number
\newcommand{\increase}{\addtocounter{pppp}{1}} %problem number

% Arguments: Title, Number of Points
\newcommand{\newproblem}[2]{
  \ifnum\me=0
    \ifnum\prob>0 \newpage \fi
    \increase
    \setcounter{page}{1}
    \handout{\name{} (\netid), Homework \num, Problem \arabic{pppp}}
    {\today}{Name: \name{} (\netid)}{Due: \due}
    {Solutions to Problem \prob\ of Homework \num\ (#2)}
  \else
    \increase
    \section*{Problem \num-\prob~(#1) \hfill {#2}}
  \fi
}

% \newcommand{\newproblem}[2]{\increase
% \section*{Problem \num-\prob~(#1) \hfill {#2}}
% }

\def\squarebox#1{\hbox to #1{\hfill\vbox to #1{\vfill}}}
\def\qed{\hspace*{\fill}
  \vbox{\hrule\hbox{\vrule\squarebox{.667em}\vrule}\hrule}}
\newenvironment{solution}{\begin{trivlist}\item[]{\bf Solution:}}
  {\qed \end{trivlist}}
\newenvironment{solsketch}{\begin{trivlist}\item[]{\bf Solution
      Sketch:}} {\qed \end{trivlist}}
\newenvironment{code}{\begin{tabbing}
    12345\=12345\=12345\=12345\=12345\=12345\=12345\=12345\= \kill }
  {\end{tabbing}}

\newcommand{\eqref}[1]{Equation~(\ref{eq:#1})}

\newcommand{\hint}[1]{({\bf Hint}: {#1})}
% Put more macros here, as needed.
\newcommand{\room}{\medskip\ni}
\newcommand{\brak}[1]{\langle #1 \rangle}
\newcommand{\bit}[1]{\{0,1\}^{#1}}
\newcommand{\zo}{\{0,1\}}
\newcommand{\C}{{\cal C}}

\newcommand{\nin}{\not\in}
\newcommand{\set}[1]{\{#1\}}
\renewcommand{\ni}{\noindent}
\renewcommand{\gets}{\leftarrow}
\renewcommand{\to}{\rightarrow}
\newcommand{\assign}{:=}

\newcommand{\AND}{\wedge}
\newcommand{\OR}{\vee}

\newcommand{\For}{\mbox{\bf for }}
\newcommand{\To}{\mbox{\bf to }}
\newcommand{\Do}{\mbox{\bf do }}
\newcommand{\If}{\mbox{\bf if }}
\newcommand{\Then}{\mbox{\bf then }}
\newcommand{\Else}{\mbox{\bf else }}
\newcommand{\While}{\mbox{\bf while }}
\newcommand{\Repeat}{\mbox{\bf repeat }}
\newcommand{\Until}{\mbox{\bf until }}
\newcommand{\Return}{\mbox{\bf return }}
\newcommand{\Halt}{\mbox{\bf halt }}
\newcommand{\Swap}{\mbox{\bf swap }}
\newcommand{\Ex}[2]{\textrm{exchange } #1 \textrm{ with } #2}



\begin{document}

\ifnum\me=0

% Collaborators (on a per task basis):
%
% Task 1: *********** INSERT COLLABORATORS HERE *********** 
% Task 2: *********** INSERT COLLABORATORS HERE *********** 
% etc.
%

\fi

\ifnum\me=1

\handout{PS \num}{\today}{Lecturer: Yevgeniy Dodis}{Due: \due}
{Solution {\em Sketches} to Problem Set \num}

\fi

\ifnum\me=2

\handout{PS \num}{\today}{Lecturer: Yevgeniy Dodis}{Due: \due}{Problem
  Set \num}

\fi



\newproblem{Asymptotic Comparisons}{10 Points}

\noindent
For each of the following pairs of functions $f(n)$ and $g(n)$, state
whether $f$ is $O(g)$; whether $f$ is $o(g)$; whether $f$ is
$\Theta(g)$; whether $f$ is $\Omega(g)$; and whether $f$ is
$\omega(g)$.  (More than one of these can be true for a single pair!)

\begin{itemize}
  \item[(a)] $f(n) = n!$; \quad
  $g(n) = (n+1)!$.

  \ifnum\me<2
\begin{solution}   INSERT YOUR SOLUTION HERE   \end{solution}
  \fi

   \item[(b)] $f(n) =n^{\log_2 c}$; \quad $g(n) = c^{\log_2 n}$.

  \ifnum\me<2
\begin{solution}   INSERT YOUR SOLUTION HERE   \end{solution}
  \fi

  \item[(c)] $f(n) = \log(n^5+5n^4+4n^3+3n^2+2n+1)$; \quad
  $g(n) = \log(n^{10}+n^8+n^6+n^4+n^2+100$.

  \ifnum\me<2
\begin{solution}   INSERT YOUR SOLUTION HERE   \end{solution}
  \fi

  \item[(d)] $f(n) = \frac{3^n}{n+n\log n}$; \quad
  $g(n) =9^{\sqrt{n}}$.

  \ifnum\me<2
\begin{solution}   INSERT YOUR SOLUTION HERE  \end{solution}
  \fi

  \item[(e)] $f(n) = 2^{\sqrt{\log_2 n}}$; \quad $g(n)=n^{1/3}$.

  \ifnum\me<2
\begin{solution}   INSERT YOUR SOLUTION HERE   \end{solution}
  \fi

\end{itemize}

\newproblem{Order of Growth}{12 Points}

\begin{itemize}
  \item[(a)] (8 Points) For each of the following functions $f(n)$,
  find a ``canonical'' function\footnote{I.e., function of the form
    $a^n n^b \log^c n$ for constants $a\ge 1$ and $b,c\ge 0$.}  $g(n)$
  such that $f(n) = \Theta(g(n))$. For example,
  $3200n + 2n^2 \log^{24} (n^2) = \Theta(n^2\log^{24} n)$. Briefly
  justify your answers (and I mean {\em briefly}).
\end{itemize}
$$3^n + 7 n^{73},~~3^{n+\log_3 n},~~\log(n^{34} + 5),~~2^{2n},
~~\sqrt{n^5+30n^4},~~~ \frac{n^4 - 5n}{55555},~~\log^2 n + 33,~~n^2
\log^3 n + n^3 \log^2 n$$

\ifnum\me<2
\begin{solution}   INSERT YOUR SOLUTION HERE   \end{solution}
\fi
\begin{itemize}
  \item[(b)] (4 Points) Based on your answers in part (a), sort the
  resulting ``canonical'' (not original)\footnote{Thsi way even if you
    got part (a) wrong, you can still have correct solution to part
    (b).}  functions in asymptotically increasing order.
\end{itemize}
\ifnum\me<2
\begin{solution}   INSERT YOUR SOLUTION HERE   \end{solution}
\fi



\newproblem{Insertion Sort and Inversions}{16 points}

\noindent
Let $A[1, \ldots, n]$ be an array of $n$ distinct numbers.  If $i < j$
and $A[i] > A[j]$, then the pair $(i,j)$ is called an \emph{inversion}
of $A$.

\begin{itemize}

  \item[(a)] (2 points) List all inversions of the array
  $\langle 8,5,2,7,9 \rangle$.

  \ifnum\me<2
  \begin{solution}
    INSERT YOUR SOLUTION HERE
  \end{solution}
  \fi

  \item[(b)] (3 points) Which arrays with distinct elements from the
  set $\{1,2,\ldots, n\}$ have the smallest and the largest number of
  inversions and why?  State the expressions {\em exactly} in terms of
  $n$.

  \ifnum\me<2
  \begin{solution}
    INSERT YOUR SOLUTION HERE
  \end{solution}
  \fi

  \item[(c)] (5 points) What is the relationship between the running
  time of {\sc{Insertion-Sort}} and the number of inversions $I$ in
  the input array? {\em Justify your answer}.

  \ifnum\me<2
  \begin{solution}
   INSERT YOUR SOLUTION HERE
  \end{solution}
  \fi
  
  \item[(d)] (3 points) For any $0<a<1/2$, construct an array
  for which insertion sort has a run time of $an^2+\Theta(n)$.
   \ifnum\me<2
  \begin{solution}
    INSERT YOUR SOLUTION HERE
  \end{solution}
  \fi
\item[(e)] (3 points) Let $A[1, \ldots, n]$
  be a random permutation of $\{1,2,\ldots, n\}$.  What is the
  expected number of inversions of $A$.  What can you conclude about
  the average case running time of {\sc{Insertion-Sort}} (where the
  average is over all arrays $A$ of size $n$)?
 
 \textbf{Hint:} Recall the linearity of expectation, i.e., for any
  real $a, b, c$ and any random variables $X, Y$,
  \[E(aX+bY+c) = a E(X) + b E(Y) + c \; .\]
  \ifnum\me<2
  \begin{solution}
    INSERT YOUR SOLUTION HERE
  \end{solution}
  \fi

\end{itemize}



\end{document}


